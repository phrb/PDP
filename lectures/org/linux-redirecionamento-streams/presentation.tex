% Created 2021-06-18 Fri 12:04
% Intended LaTeX compiler: pdflatex
\documentclass[10pt, compress, aspectratio=169, xcolor={table,usenames,dvipsnames}]{beamer}

\usepackage{booktabs}
\mode<beamer>{\usetheme[numbering=fraction, progressbar=none, titleformat frame=regular, titleformat title=regular, sectionpage=progressbar]{metropolis}}
\usepackage{booktabs}
\usepackage{array}
\usepackage{multirow}
\usepackage{caption}
\usepackage{graphicx}
\usepackage[english]{babel}
\usepackage[scale=2]{ccicons}
\usepackage{hyperref}
\usepackage{relsize}
\usepackage{amsmath}
\usepackage{bm}
\usepackage{ragged2e}
\usepackage{textcomp}
\usepackage{pgfplots}
\usepgfplotslibrary{dateplot}
\definecolor{Base}{HTML}{191F26}
\colorlet{Accent}{BrickRed}
\colorlet{CodeBg}{Gray!20}
\colorlet{CodeHighBg}{Accent!10}
\colorlet{Highlight}{Accent!18}
\setbeamercolor{alerted text}{fg=Accent}
\setbeamercolor{frametitle}{fg=Accent,bg=normal text.bg}
\setbeamercolor{normal text}{bg=black!2,fg=Base}
\usefonttheme{professionalfonts}
\usepackage{newpxtext}
\usepackage{newpxmath}
\usepackage{minted}
\usemintedstyle{vs}
\setminted{
frame=none,
bgcolor=CodeBg,
fontsize=\footnotesize,
baselinestretch=1.2,
framesep=0.6em,
tabsize=2,
breaklines
}
\AtBeginEnvironment{snugshade*}{\vspace{-\FrameSep}}
\AfterEndEnvironment{snugshade*}{\vspace{-\FrameSep}}
\usepackage{fancyvrb}
\usepackage{fvextra}
\RecustomVerbatimEnvironment{verbatim}{Verbatim}{fontsize=\footnotesize,
breaklines=true,
frame=single,
framerule=1mm,
rulecolor=CodeBg}
\usepackage{DejaVuSansMono}
\setmonofont{DejaVuSansMono}
\renewcommand*{\UrlFont}{\ttfamily\smaller[2]\relax}
\addtobeamertemplate{block begin}{}{\justifying}
\captionsetup[figure]{labelformat=empty}
\hypersetup{
colorlinks=true,
linkcolor={Accent},
citecolor={Accent},
urlcolor={Accent}
}
\makeatletter
\setlength{\metropolis@titleseparator@linewidth}{1pt}
\setlength{\metropolis@progressonsectionpage@linewidth}{2.5pt}
\makeatother
\usetheme{default}
\author{\footnotesize Pedro Bruel \newline \scriptsize \emph{pedro.bruel@gmail.com}}
\date{\scriptsize 18 de Junho de 2021}
\title{ Redirecionamento de Streams \\
Padrão em Sistemas POSIX}
\begin{document}

\maketitle

\section{Introdução}
\label{sec:org4f3bd96}
\begin{frame}[label={sec:org941f0e5}]{Conteúdo e Objetivos de Aprendizagem}
\begin{columns}
\begin{column}{0.65\columnwidth}
\begin{block}{Tópicos}
\begin{itemize}
\item \alert{Streams}, ou \alert{fluxos} padrão em Processos POSIX,
chamadas de sistema \emph{\alert{dup}} e \emph{\alert{dup2}}
\end{itemize}

\begin{block}{Habilidades}
\begin{itemize}
\item Escrever scripts em \alert{bash} e programas  em \alert{C} que \alert{redirecionem} os \alert{streams}
de \alert{Entrada/Saída} de programas
\end{itemize}
\end{block}

\begin{block}{Conhecimentos}
\begin{itemize}
\item Compreender o redirecionamento de streams  de Entrada/Saída como a \alert{duplicação
de descritores de arquivo} em processos POSIX
\end{itemize}
\end{block}
\end{block}
\end{column}

\begin{column}{0.35\columnwidth}
\begin{center}
\includegraphics[width=0.7\textwidth]{../../ppd-images/tux_badge.png}
\end{center}
\end{column}
\end{columns}
\end{frame}
\begin{frame}[label={sec:orgc9fc546}]{Recursos Extras}
\begin{columns}
\begin{column}{0.5\columnwidth}
\begin{block}{The Linux API}
\begin{center}
\includegraphics[width=0.5\columnwidth]{../../ppd-images/kerrisk_api.png}
\end{center}

\begin{itemize}
\item \url{https://man7.org/tlpi}
\item Capítulos 4, 5, e 44
\end{itemize}
\end{block}
\end{column}

\begin{column}{0.5\columnwidth}
\begin{block}{Site com Slides e Exemplos}
\begin{center}
\includegraphics[width=\columnwidth]{../../ppd-images/streams_redir.png}
\end{center}

\begin{itemize}
\item \url{https://phrb.github.io/PPD}
\end{itemize}
\end{block}
\end{column}
\end{columns}
\end{frame}

\begin{frame}[label={sec:org48cbc37}]{Registrando Mensagens e Monitorando Sistemas}
\begin{columns}
\begin{column}{0.6\columnwidth}
\begin{block}{Problemas}
\begin{itemize}
\item Em sistemas POSIX,  quais são as formas de \alert{registrar  as mensagens impressas}
por um programa durante sua execução?
\item No Linux, quais as formas de \alert{monitorar  o uso de recursos} como CPU, memória,
entrada/saída, e rede?
\end{itemize}
\end{block}
\end{column}
\begin{column}{0.4\columnwidth}
\begin{center}
  \includegraphics[height=1.3cm]{../../ppd-images/document_icon.png}
  \includegraphics[height=1.3cm]{../../ppd-images/document_icon_2.png}

  \vspace{1em}

  \includegraphics[height=1.3cm]{../../ppd-images/monitoring_icon.png}
\end{center}
\end{column}
\end{columns}
\end{frame}
\begin{frame}[label={sec:org9f1af88},t,fragile]{Registrando Mensagens e Monitorando Sistemas}
\begin{columns}
\begin{column}{0.5\columnwidth}
\begin{block}{Registro de Mensagens}
\begin{itemize}
\item
\end{itemize}
\end{block}
\end{column}

\begin{column}{0.5\columnwidth}
\begin{block}{Monitoramento de Recursos}
\begin{itemize}
\item
\end{itemize}
\end{block}
\end{column}
\end{columns}
\end{frame}

\section{Redirecionamento de Entrada/Saída em \emph{bash}}
\label{sec:orga20128a}
\begin{frame}[label={sec:orgd58ef0c}]{Streams Padrão em Processos POSIX}
\begin{itemize}
\item \alert{Streams}, ou \alert{fluxos}, são \alert{canais de comunicação} indexados por \alert{descritores
de arquivo}
\item Todo processo POSIX abre os seguintes \alert{streams} por padrão:
\end{itemize}
\begin{center}
\small
\begin{tabular}{clll}
\toprule
\textbf{Descritor} & \textbf{Propósito} & \textbf{Nome POSIX} &  \textbf{Stream \textit{stdio}}\\
\midrule
0 & \alert{Entrada} padrão & STDIN\_FILENO & \emph{stdin}\\
1 & \alert{Saída} padrão & STDOUT\_FILENO & \emph{stdout}\\
2 & \alert{Erro} padrão & STDERR\_FILENO & \emph{stderr}\\
\bottomrule
\end{tabular}
\end{center}
\end{frame}
\begin{frame}[label={sec:org680cc06},fragile]{Usando Arquivos de \emph{Log}: Redirecionamento de Saída e Erros}
 \begin{figure}
\begin{minipage}{.8\textwidth}

\begin{minted}[]{bash}
date > existe.txt && cat existe.txt
\end{minted}

\begin{verbatim}
Wed Jun 16 05:38:58 PM -03 2021
\end{verbatim}


\vspace{-0.4em}
\pause

\begin{itemize}
\item Redirecionando \alert{erro} e \alert{saída} para arquivos diferentes:
\end{itemize}

\vspace{0.3em}

\begin{minted}[]{bash}
cat existe.txt não-existe.txt 1> saída.log 2> erros.log
\end{minted}

\vspace{0.3em}
\pause

\begin{itemize}
\item Verificando os \alert{logs}:
\end{itemize}

\vspace{0.3em}

\begin{minted}[]{bash}
cat saída.log
\end{minted}

\begin{verbatim}
Wed Jun 16 05:38:58 PM -03 2021
\end{verbatim}


\pause

\begin{minted}[]{bash}
cat erros.log
\end{minted}

\begin{verbatim}
cat: não-existe.txt: No such file or directory
\end{verbatim}


\end{minipage}
\end{figure}
\end{frame}

\begin{frame}[label={sec:org1ae8c6a},fragile]{Monitorando o Linux: Redirecionamento de Saída}
 \begin{block}{Monitorando o Consumo de Memória}
\begin{figure}
\begin{minipage}{\textwidth}
\begin{minted}[]{bash}
free -h > mem.log && cat mem.log
\end{minted}

\pause

\begin{verbatim}
               total        used        free      shared  buff/cache   available
Mem:            31Gi       7.2Gi       1.1Gi       1.4Gi        22Gi        22Gi
Swap:             0B          0B          0B
\end{verbatim}

\pause

\begin{minted}[]{bash}
cat /proc/meminfo > meminfo.log
\end{minted}

\end{minipage}
\end{figure}
\end{block}
\end{frame}

\begin{frame}[label={sec:org3441fad},fragile]{Monitorando o Linux: Redirecionamento de Entrada}
 \begin{block}{Monitorando o Clock da CPU}
\begin{figure}
\begin{minipage}{0.9\textwidth}
\begin{minted}[]{bash}
grep "cpu MHz" < /proc/cpuinfo > cpu_clock.log && cat cpu_clock.log
\end{minted}

\pause

\begin{verbatim}
cpu MHz        : 3000.000
cpu MHz        : 1746.784
cpu MHz        : 3000.000
cpu MHz        : 3000.000
cpu MHz        : 3000.000
cpu MHz        : 3000.000
cpu MHz        : 900.006
cpu MHz        : 900.001
\end{verbatim}


\end{minipage}
\end{figure}

\pause
\end{block}

\begin{block}{Mais Exemplos}
\begin{itemize}
\item \url{https://www.gnu.org/software/bash/manual/html\_node/Redirections.html}
\end{itemize}
\end{block}
\end{frame}

\begin{frame}[label={sec:org0f9aebc},fragile]{Exercício: Script de Redirecionamento}
 Escreva um \alert{script \emph{bash}} que:

\begin{enumerate}
\item Receba \alert{3 argumentos}:
\begin{itemize}
\item \texttt{\$1}: Um programa e seus argumentos
\item \texttt{\$2}: Um arquivo de entrada
\item \texttt{\$3}: Um arquivo de saída
\end{itemize}
\item \alert{Execute} o programa \texttt{\$1} com stream de \alert{Entrada} \texttt{\$2}, e redirecione \alert{Saída
e Erro} para \texttt{\$3}
\end{enumerate}


\pause

\begin{figure}
\begin{minipage}{.43\textwidth}
\begin{minted}[]{bash}
#!/usr/bin/bash
/usr/bin/bash -c "$1 < $2 &> $3"
\end{minted}

\end{minipage}
\end{figure}
\end{frame}

\begin{frame}[label={sec:org1cd38f0},fragile]{Exercício: Script de Redirecionamento}
 \begin{figure}
\begin{minipage}{\textwidth}
\begin{minted}[]{bash}
#!/usr/bin/bash
/usr/bin/bash -c "$1 < $2 &> $3"
\end{minted}

\begin{minted}[]{bash}
./src/bash_example/log.sh 'grep "cpu MHz"' /proc/cpuinfo out.log && cat out.log
\end{minted}

\begin{verbatim}
cpu MHz        : 1303.327
cpu MHz        : 1682.174
cpu MHz        : 3000.000
cpu MHz        : 3000.000
cpu MHz        : 3000.000
cpu MHz        : 3000.000
cpu MHz        : 3000.000
cpu MHz        : 3000.000
\end{verbatim}


\end{minipage}
\end{figure}
\end{frame}

\begin{frame}[label={sec:orgcd1d6b4},fragile]{Síntese: Redirecionamento de Streams em \emph{bash}}
\begin{figure}
\small
  \begin{tabular}{p{0.2\textwidth}p{0.52\textwidth}}
    \toprule
    \textbf{Sintaxe} & \textbf{Efeito} \\
    \midrule
    \begin{minipage}[t]{0.3\textwidth}
\texttt{[n]> arquivo} \\
\texttt{[n]>> arquivo}
    \end{minipage} & Redireciona o descritor \texttt{n} para \texttt{arquivo},
    sobrescrevendo (\texttt{>}) ou adicionando (\texttt{>>}) \\
    \addlinespace[1em]
    \begin{minipage}[t]{0.3\textwidth}
\texttt{\&> arquivo}
    \end{minipage}
    & Redireciona \alert{Erro} e \alert{Saída} Padrão para \texttt{arquivo} \\
    \addlinespace[1em]
    \begin{minipage}[c]{0.3\textwidth}
\texttt{> arquivo 2>\&1}
    \end{minipage}
    & Redireciona \alert{Saída Padrão} (\texttt{fd=1}) para \texttt{arquivo},
    e rediciona \alert{Erro Padrão} (\texttt{fd=2}) para \alert{Saída Padrão} \\
    \addlinespace[1em]
    \begin{minipage}[c]{0.3\textwidth}
\texttt{< arquivo}
    \end{minipage}
    & Redireciona \alert{Entrada Padrão} para ler de \texttt{arquivo} \\
    \addlinespace[0.55em]\bottomrule
  \end{tabular}
\end{figure}
\end{frame}

\begin{frame}[label={sec:org5a91470}]{Conectando Streams de Entrada/Saída}
\begin{columns}
\begin{column}{0.6\columnwidth}
\begin{block}{Problemas}
\begin{itemize}
\item Como \alert{conectar os streams} de entrada/saída de dois processos através do bash?
\item E através de um \alert{programa em C}?
\end{itemize}
\end{block}
\end{column}
\begin{column}{0.4\columnwidth}
\begin{center}
  \includegraphics[height=1.3cm]{../../ppd-images/chain_icon.png}
  \includegraphics[height=1.3cm]{../../ppd-images/pipe_icon.png}

  \vspace{1em}

  \includegraphics[height=1.3cm]{../../ppd-images/brick_icon.png}
\end{center}
\end{column}
\end{columns}
\end{frame}
\begin{frame}[label={sec:org9309a53},t,fragile]{Conectando Streams de Entrada/Saída}
\begin{columns}
\begin{column}{0.5\columnwidth}
\begin{block}{Conectando Streams em \emph{bash}}
\begin{itemize}
\item
\end{itemize}
\end{block}
\end{column}

\begin{column}{0.5\columnwidth}
\begin{block}{Conectando Streams em \emph{C}}
\begin{itemize}
\item
\end{itemize}
\end{block}
\end{column}
\end{columns}
\end{frame}

\section{Duplicando Descritores de Arquivo com Chamadas de Sistema}
\label{sec:orgef53343}
\begin{frame}[label={sec:org0c52885}]{Descritores de Arquivo em Processos POSIX}
\begin{center}
\includegraphics[width=0.68\textwidth]{../../ppd-images/file_descriptors_kerrisk.pdf}
\end{center}

\begin{center}
\scriptsize
The Linux Programming API, Michael Kerrisk, pág. 95
\end{center}
\end{frame}

\begin{frame}[label={sec:orgfb9406e},fragile]{Algumas Chamadas POSIX}
 \begin{figure}
\begin{minipage}{0.6\textwidth}

\begin{minted}[,fontsize=\scriptsize]{c}
#include <sys/stat.h>
#include <fcntl.h>
#include <unistd.h>

int open(const char *pathname, int flags, ...);
int close(int fd);
ssize_t read(int fd, void *buffer, size_t count);
ssize_t write(int fd, void *buffer, size_t count);

pid_t fork(void);
int execlp(const char *filename, const char *arg, ...);
\end{minted}

\pause
\vspace{0.3em}

\begin{minted}[,fontsize=\scriptsize,bgcolor=CodeHighBg]{c}
int dup(int oldfd);
int dup2(int oldfd, int newfd);
int pipe(int filedes[2]);
\end{minted}

\end{minipage}
\end{figure}
\end{frame}

\begin{frame}[label={sec:orgec4e9c6},fragile]{Chamadas \emph{dup} e \emph{dup2}}
 \texttt{int dup(int oldfd);}
\begin{itemize}
\item \alert{Duplica} o descritor de arquivo \texttt{oldfd}, usando o menor descritor disponível
\end{itemize}

\texttt{int dup2(int oldfd, int newfd);}
\begin{itemize}
\item \alert{Duplica} \texttt{oldfd} usando \texttt{newfd}, fecha \texttt{newfd} se necessário
\end{itemize}

\begin{figure}
\begin{minipage}{.95\textwidth}

\begin{minted}[,fontsize=\scriptsize]{bash}
man dup | grep "NAME" -A 10
\end{minted}

\begin{verbatim}[fontsize=\scriptsize]
NAME
  dup, dup2 — duplicate an open file descriptor

SYNOPSIS
  include <unistd.h>
  int dup(int fildes);
  int dup2(int fildes, int fildes2);

DESCRIPTION
  The  dup()  function provides an alternative interface to the service provided [...]
\end{verbatim}

\end{minipage}
\end{figure}
\end{frame}

\begin{frame}[label={sec:org3d1ca01},fragile]{Chamada \emph{pipe}}
 \texttt{int pipe(int filedes[2]);}
\begin{itemize}
\item Abre  um  canal  de  comunicação \alert{entre  processos},  usando  \alert{descritores  de
arquivo}
\end{itemize}

\begin{figure}
\begin{minipage}{.95\textwidth}

\begin{minted}[,fontsize=\scriptsize]{bash}
man pipe | grep "NAME" -A 10
\end{minted}

\begin{verbatim}[fontsize=\scriptsize]
NAME
  pipe — create an interprocess channel

SYNOPSIS
  include <unistd.h>
  int pipe(int fildes[2]);

DESCRIPTION
  The pipe() function shall create a pipe and place two file descriptors, one each into
  the arguments fildes[0] and fildes[1], that refer to the open file  descriptions [..]
\end{verbatim}

\end{minipage}
\end{figure}
\end{frame}

\begin{frame}[label={sec:org3872299},fragile]{Exercício: Redirecionamento em C}
 \begin{columns}
\begin{column}{0.5\columnwidth}
\begin{block}{Usando um \emph{pipe} para Conectar Filtros}
\begin{center}
\includegraphics[width=\textwidth]{../../ppd-images/pipe_filters.pdf}
\end{center}

\begin{center}
\scriptsize
The Linux Programming API, Michael Kerrisk, cap. 44, pág. 890
\end{center}
\end{block}
\end{column}

\begin{column}{0.3\columnwidth}
\begin{block}{Exemplo em \emph{C}}
\begin{figure}
\begin{minipage}{\textwidth}
\begin{minted}[]{bash}
tree src/pipe_example
\end{minted}

\begin{verbatim}
src/pipe_example
├── Makefile
├── pipe_example.md
├── pipe_example.org
└── pipe_ls_wc.c

0 directories, 4 files
\end{verbatim}


\end{minipage}
\end{figure}
\end{block}
\end{column}
\end{columns}

\begin{block}{Código Fonte}
\begin{itemize}
\item The Linux Programming API, Michael Kerrisk, cap. 44, pág. 890
\item \url{https://man7.org/tlpi/code/online/dist/pipes/pipe\_ls\_wc.c.html}
\end{itemize}
\end{block}
\end{frame}

\begin{frame}[label={sec:org96e2b27},fragile]{Síntese: Duplicando Descritores de Arquivo}
 \begin{columns}
\begin{column}{0.5\columnwidth}
\begin{block}{Descritores de Arquivo}
\begin{center}
\includegraphics[width=\textwidth]{../../ppd-images/file_descriptors_kerrisk.pdf}
\end{center}
\end{block}
\end{column}

\begin{column}{0.5\columnwidth}
\begin{block}{Pipes}
\begin{center}
\includegraphics[width=\textwidth]{../../ppd-images/pipe_filters.pdf}
\end{center}

\vspace{-0.9em}

\begin{block}{Chamadas de Sistema}
\vspace{0.2em}

\bgroup\scriptsize
\texttt{int dup(int oldfd);}
\vspace{-0.6em}
\begin{itemize}
\item \alert{Duplica} o descritor de arquivo \texttt{oldfd}, usando o menor descritor disponível
\end{itemize}

\vspace{-0.6em}
\texttt{int dup2(int oldfd, int newfd);}
\vspace{-0.6em}
\begin{itemize}
\item \alert{Duplica} \texttt{oldfd} usando \texttt{newfd}, fecha \texttt{newfd} se necessário
\end{itemize}

\vspace{-0.6em}
\texttt{int pipe(int filedes[2]);}
\vspace{-0.6em}
\begin{itemize}
\item Abre  um  canal  de  comunicação \alert{entre  processos},  usando  \alert{descritores  de
arquivo}
\end{itemize}
\egroup
\end{block}
\end{block}
\end{column}
\end{columns}
\end{frame}

\section{Revisitando os Objetivos de Aprendizagem}
\label{sec:orgdda8965}
\begin{frame}[label={sec:org9cae99f}]{Objetivos de Aprendizagem e Exercício}
\vspace{-0.8em}
\begin{columns}
\begin{column}{0.65\columnwidth}
\begin{block}{Tópicos}
\begin{itemize}
\item \alert{Streams}, ou \alert{fluxos} padrão em Processos POSIX,
chamadas de sistema \emph{\alert{dup}} e \emph{\alert{dup2}}
\end{itemize}

\vspace{-0.5em}

\begin{block}{Objetivos}
\begin{itemize}
\item Escrever scripts em \alert{bash} e programas  em \alert{C} que \alert{redirecionem} os \alert{streams}
de \alert{Entrada/Saída} de programas
\item Compreender o redirecionamento de streams  de Entrada/Saída como a \alert{duplicação
de descritores de arquivo} em processos POSIX
\end{itemize}

\vspace{-0.5em}
\end{block}

\begin{block}{Exercício}
\begin{itemize}
\item Pesquise sobre a chamada de sistema  \emph{\alert{fcntl}}, \alert{escreva} as chamadas \emph{dup} e
\emph{dup2} com ela, e as \alert{substitua} no exemplo da conexão de filtros com \emph{pipe}
\end{itemize}
\end{block}
\end{block}
\end{column}
\begin{column}{0.35\columnwidth}
\begin{center}
\includegraphics[width=0.7\textwidth]{../../ppd-images/tux_badge.png}
\end{center}
\end{column}
\end{columns}
\end{frame}


\maketitle
\end{document}
