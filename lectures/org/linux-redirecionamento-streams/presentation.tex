% Created 2021-06-15 Tue 17:56
% Intended LaTeX compiler: pdflatex
\documentclass[10pt, compress, aspectratio=169, xcolor={table,usenames,dvipsnames}]{beamer}

\usepackage{booktabs}
\mode<beamer>{\usetheme[numbering=fraction, progressbar=none, titleformat frame=regular, titleformat title=regular, sectionpage=progressbar]{metropolis}}
\usepackage{booktabs}
\usepackage{array}
\usepackage{multirow}
\usepackage{caption}
\usepackage{graphicx}
\usepackage[english]{babel}
\usepackage[scale=2]{ccicons}
\usepackage{hyperref}
\usepackage{relsize}
\usepackage{amsmath}
\usepackage{bm}
\usepackage{ragged2e}
\usepackage{textcomp}
\usepackage{pgfplots}
\usepgfplotslibrary{dateplot}
\definecolor{Base}{HTML}{191F26}
\colorlet{Accent}{BrickRed}
\colorlet{Highlight}{Accent!18}
\setbeamercolor{alerted text}{fg=Accent}
\setbeamercolor{frametitle}{fg=Accent,bg=normal text.bg}
\setbeamercolor{normal text}{bg=black!2,fg=Base}
\usefonttheme{professionalfonts}
\usepackage{newpxtext}
\usepackage{newpxmath}
\usepackage{minted}
\usemintedstyle{vs}
\setminted{frame=lines,bgcolor=Black!9,fontsize=\small,baselinestretch=1.2,framesep=0.5em}
\usepackage{DejaVuSansMono}
\setmonofont{DejaVuSansMono}
\renewcommand*{\UrlFont}{\ttfamily\smaller[2]\relax}
\addtobeamertemplate{block begin}{}{\justifying}
\captionsetup[figure]{labelformat=empty}
\hypersetup{
colorlinks=true,
linkcolor={Accent},
citecolor={Accent},
urlcolor={Accent}
}
\makeatletter
\setlength{\metropolis@titleseparator@linewidth}{1pt}
\setlength{\metropolis@progressonsectionpage@linewidth}{2.5pt}
\makeatother
\usetheme{default}
\author{\footnotesize Pedro Bruel \newline \scriptsize \emph{pedro.bruel@gmail.com}}
\date{\scriptsize 18 de Junho de 2021}
\title{ Redirecionamento de Entrada/Saída \\
padrão em sistemas POSIX}
\begin{document}

\maketitle

\section{Introdução}
\label{sec:org8330c1e}
\begin{frame}[label={sec:org9041c78}]{Objetivos da Aula e Recursos Extras}
\end{frame}
\begin{frame}[label={sec:orgb3d202a}]{Revisão: Algumas Chamadas POSIX para Arquivos}
\end{frame}
\begin{frame}[label={sec:org179061a}]{Registrando Mensagens e Monitorando Sistemas}
\begin{columns}
\begin{column}{0.6\columnwidth}
\begin{block}{Motivação}
\begin{itemize}
\item Em sistemas POSIX,  quais são as formas de \alert{registrar  as mensagens impressas}
por um programa durante sua execução?
\item No Linux, quais as formas de \alert{monitorar  o uso de recursos} como CPU, memória,
entrada/saída, e rede?
\end{itemize}
\end{block}
\end{column}
\begin{column}{0.4\columnwidth}
\begin{center}
  \includegraphics[height=1.3cm]{../../ppd-images/document_icon.png}
  \includegraphics[height=1.3cm]{../../ppd-images/document_icon_2.png}

  \vspace{1em}

  \includegraphics[height=1.3cm]{../../ppd-images/monitoring_icon.png}
\end{center}
\end{column}
\end{columns}
\end{frame}
\begin{frame}[label={sec:orgab23e77}]{Descritores de Arquivos em Processos POSIX}
\begin{center}
\includegraphics[width=0.68\textwidth]{../../ppd-images/file_descriptors_kerrisk.pdf}
\end{center}

\begin{center}
\scriptsize
The Linux Programming API, Michael Kerrisk, pág. 95
\end{center}
\end{frame}

\begin{frame}[label={sec:org3263c57}]{Streams Padrão em Processos POSIX}
\begin{center}
\small
\begin{tabular}{clll}
\toprule
\textbf{Descritor} & \textbf{Propósito} & \textbf{Nome POSIX} &  \textbf{Stream \textit{stdio}}\\
\midrule
0 & Entrada padrão & STDIN\_FILENO & \emph{stdin}\\
1 & Saída padrão & STDOUT\_FILENO & \emph{stdout}\\
2 & Erro padrão & STDERR\_FILENO & \emph{stderr}\\
\bottomrule
\end{tabular}
\end{center}
\end{frame}

\begin{frame}[label={sec:orgac3b82b},fragile]{Exemplos em \emph{bash}}
 \begin{itemize}
\item \url{https://www.gnu.org/software/bash/manual/html\_node/Redirections.html}
\end{itemize}

\begin{figure}
\begin{minipage}{0.6\textwidth}
\begin{minted}[]{bash}
head < /proc/cpuinfo > cpu_info.log
\end{minted}
\end{minipage}
\end{figure}
\end{frame}

\maketitle
\end{document}
