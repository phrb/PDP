\documentclass[10pt, compress, aspectratio=169, xcolor={table,usenames,dvipsnames}]{beamer}
\usetheme[numbering=fraction, progressbar=none, titleformat=smallcaps, sectionpage=none]{metropolis}

\usepackage{sourcecodepro}
\usepackage{booktabs}
\usepackage{array}
\usepackage{listings}
\usepackage{graphicx}
\usepackage[brazilian]{babel}
\usepackage[scale=2]{ccicons}
\usepackage{url}
\usepackage{relsize}
\usepackage{wasysym}

\setsansfont[BoldFont={Source Sans Pro Semibold},
              Numbers={OldStyle}]{Source Sans Pro}

\usepackage{pgfplots}
\usepgfplotslibrary{dateplot}

\definecolor{Base}{HTML}{191F26}
\definecolor{Highlight}{HTML}{ffda99}
\definecolor{Accent}{HTML}{bb0300}

\setbeamercolor{alerted text}{fg=Accent}
\setbeamercolor{frametitle}{fg=Base,bg=White}
\setbeamercolor{normal text}{bg=black!2,fg=Base}

\lstset{ %
  backgroundcolor={},
  basicstyle=\ttfamily\footnotesize,
  breakatwhitespace=true,
  breaklines=true,
  captionpos=n,
  commentstyle=\color{orange},
  escapeinside={\%*}{*)},
  extendedchars=true,
  frame=n,
  keywordstyle=\color{Accent},
  language=C++,
  rulecolor=\color{black},
  showspaces=false,
  showstringspaces=false,
  showtabs=false,
  stepnumber=2,
  stringstyle=\color{gray},
  tabsize=2,
  keywords={thrust,plus,device_vector, copy,transform,begin,end, copyin,
  copyout, acc, \_\_global\_\_, void, int, float, main, threadIdx, blockIdx,
  blockDim, if, else, malloc, NULL, cudaMalloc, cudaMemcpy, cudaSuccess,
  cudaGetLastError, cudaDeviceSynchronize, cudaFree, cudaMemcpyDeviceToHost,
  cudaMemcpyHostToDevice, const, data, independent, kernels, loop,
  fprintf, stderr, cudaGetErrorString, EXIT_FAILURE, for, dim3},
  otherkeywords={::, \#pragma, \#include, <<<,>>>, \&, \*, +, -, /, [, ], >, <}
}

\renewcommand*{\UrlFont}{\ttfamily\smaller[2]\relax}

\graphicspath{{../../img/}}

\title{Computação Reconfigurável com FPGAs}
\author{\footnotesize Pedro Bruel \\ {\scriptsize \emph{phrb@ime.usp.br}}}
\date{\scriptsize 1 de Julho de 2020}

\begin{document}

\maketitle

\section*{Introdução}

% \subsection*{Sobre}

% \begin{frame}
%     \frametitle{Sobre}
%     \begin{columns}[T,onlytextwidth]
%         \column{0.3\textwidth}
%         \begin{center}
%             \includegraphics[width=.75\textwidth]{pedro}
%
%             Pedro Bruel \\
%             \emph{\alert{phrb}@ime.usp.br} \\
%         \end{center}
%
%         \column{0.3\textwidth}
%         \begin{center}
%             \includegraphics[width=.65\textwidth]{alfredo}
%
%             Alfredo Goldman \\
%             \emph{\alert{gold}@ime.usp.br} \\
%         \end{center}
%
%         \column{0.3\textwidth}
%         \begin{center}
%             \includegraphics[width=.75\textwidth]{marcos}
%
%             Marcos Amarís \\
%             \emph{\alert{amaris}@ime.usp.br} \\
%         \end{center}
%     \end{columns}
% \end{frame}

% \subsection*{Roteiro}

% \begin{frame}
%     \frametitle{Roteiro}
%     \setbeamertemplate{section in toc}[sections numbered]
%     \tableofcontents
% \end{frame}

\begin{frame}
    \frametitle{Slides}
    \begin{center}
        \includegraphics[width=.14\textwidth]{github}
    \end{center}
    Os slides estão no \alert{GitHub}:

    \begin{itemize}
        \item \url{https://github.com/phrb/presentations/tree/master/aula-aceleradores-de-hardware}
    \end{itemize}
\end{frame}

\section{FPGAs}

\subsection{Computação Reconfigurável}

\begin{frame}
    \frametitle{Tendências em Hardware}
    \begin{center}
        \includegraphics[width=.95\textwidth]{49_years_processor_data}
    \end{center}
\end{frame}

\begin{frame}
    \frametitle{Tendência em Desempenho}
    \begin{center}
        \includegraphics[width=.95\textwidth]{top500_rmax_rpeak}
    \end{center}
\end{frame}

\begin{frame}
    \frametitle{Computação Reconfigurável}
    \begin{center}
        \includegraphics[width=.48\textwidth]{stratix10}
    \end{center}
\end{frame}

\begin{frame}
    \frametitle{Computação Reconfigurável}
    \begin{columns}
      \begin{column}{0.45\columnwidth}
        \alert{Field-Programmable} Gate Arrays (FPGAs):

        \begin{itemize}
        \item Dispositivos feitos de semicondutores
        \item Funcionalidade definível após fabricação
        \item \alert{Reconfiguráveis} mesmo após instalação
        \item Adaptáveis a diferentes aplicações
        \item \alert{Consumo eficiente de energia}
        \end{itemize}
      \end{column}

      \begin{column}{0.55\columnwidth}
        Compostas de uma matriz de elementos lógicos (\alert{Gate Array}):

        \begin{itemize}
        \item Elementos Lógicos: Portas Lógicas, Transistores (\alert{LEs})
        \item Conexões entre LEs: \alert{Interconnect}
        \item Tabelas de Verdade: \textit{Lookup Tables} (\alert{LUTs})
        \item 2D Gate Arrays: \alert{LEs} + \alert{LUTs} + \alert{Interconnect}
        \end{itemize}
      \end{column}
    \end{columns}
\end{frame}

\begin{frame}
    \frametitle{Programabilidade vs. Desempenho}
    \begin{center}
        \includegraphics[width=.7\textwidth]{fpga-tradeoff}
    \end{center}
\end{frame}

\begin{frame}
    \frametitle{Programabilidade vs. Desempenho vs. Generalidade}
    \begin{center}
        \includegraphics[width=.5\textwidth]{fpga-tradeoff-2}
    \end{center}
\end{frame}

\begin{frame}
    \frametitle{Programabilidade vs. Desempenho vs. Eficiência}
    \begin{center}
        \includegraphics[width=.5\textwidth]{fpga-tradeoff-3}
    \end{center}
\end{frame}

\begin{frame}
    \frametitle{Aplicações}
    \begin{center}
        \includegraphics[width=.8\textwidth]{fpga2020}
    \end{center}

    \vfill

    \begin{center}
        \scriptsize{Fonte: \url{https://anandtech.com/show/9321/intel-to-acquire-fpgaspecialist-altera-for-167-billion}}
    \end{center}
\end{frame}

\subsection{Arquitetura de FPGAs: Lógica Programável}

\begin{frame}
    \frametitle{Arquitetura de Memória: Lógica Programável}
    \begin{center}
        \includegraphics[width=.65\textwidth]{fusible-link}
    \end{center}

    \vfill

    \begin{center}
        \scriptsize{Fonte: Maxfield, Clive. The design warrior's guide to FPGAs: devices, tools and flows. Elsevier, 2004.}
    \end{center}
\end{frame}

\begin{frame}
    \frametitle{Arquitetura de Memória: Lógica Programável}
    \begin{center}
        \includegraphics[width=.65\textwidth]{fusible-programmed}
    \end{center}

    \vfill

    \begin{center}
        \scriptsize{Fonte: Maxfield, Clive. The design warrior's guide to FPGAs: devices, tools and flows. Elsevier, 2004.}
    \end{center}
\end{frame}

\begin{frame}
    \frametitle{Arquitetura de Memória: EPROM}
    \begin{center}
        \includegraphics[width=.65\textwidth]{eprom-transistor}
    \end{center}

    \vfill

    \begin{center}
        \scriptsize{Fonte: Maxfield, Clive. The design warrior's guide to FPGAs: devices, tools and flows. Elsevier, 2004.}
    \end{center}
\end{frame}

\begin{frame}
    \frametitle{Arquitetura de FPGAs}
    Arquitetura de Memória:
    \begin{itemize}
        \item Lógica Programável: Static Random Access Memory (\alert{SRAM})
        \item Memória: Synchronous Dynamic Random Access Memory (\alert{SDRAM})
    \end{itemize}

    \alert{2D Gate Arrays}:

    \begin{itemize}
        \item Elementos Lógicos: Portas Lógicas, Transistores (\alert{LEs})
        \item Conexões entre LEs: \alert{Interconnect}
        \item Tabelas de Verdade: \textit{Lookup Tables} (\alert{LUTs})
    \end{itemize}
\end{frame}

\begin{frame}[fragile]
    \frametitle{FPGAs: Visão Simplificada de Alto Nível}
    \begin{center}
        \includegraphics[width=.8\textwidth]{FPGA_simple}
    \end{center}

    \vfill

    \begin{center}
        \scriptsize{Fonte: Maxfield, Clive. The design warrior's guide to FPGAs: devices, tools and flows. Elsevier, 2004.}
    \end{center}
\end{frame}

\begin{frame}
    \frametitle{FPGAs: Visão Simplificada de Baixo Nível}
    \begin{center}
        \includegraphics[width=.65\textwidth]{logic-block}
    \end{center}

    \vfill

    \begin{center}
        \scriptsize{Fonte: Maxfield, Clive. The design warrior's guide to FPGAs: devices, tools and flows. Elsevier, 2004.}
    \end{center}
\end{frame}

\begin{frame}
    \frametitle{FPGAs: Visão Simplificada de Baixo Nível}
    \begin{center}
        \includegraphics[width=.65\textwidth]{lut}
    \end{center}

    \vfill

    \begin{center}
        \scriptsize{Fonte: Maxfield, Clive. The design warrior's guide to FPGAs: devices, tools and flows. Elsevier, 2004.}
    \end{center}
\end{frame}

\begin{frame}
    \frametitle{FPGAs Hoje}
    Arquitetura de FPGAs contemporânea:
    \begin{itemize}
        \item Associada a uma \alert{CPU}
        \item Static Random Access Memory (\alert{SRAM})
        \item Synchronous Dynamic Random Access Memory (\alert{SDRAM})
        \item \alert{I/O}
        \item Elementos Lógicos (Logic Elements, \alert{LEs})
        \item Interconexões
    \end{itemize}
\end{frame}

\begin{frame}
    \frametitle{FPGAs SoC: Cyclone V}
    \begin{center}
        \includegraphics[width=.8\textwidth]{cycloneV}
    \end{center}
\end{frame}

\begin{frame}
    \frametitle{Programando FPGAs}
    Hardware Description Languages (\alert{HDL}):
    \begin{itemize}
        \item Definição de \textit{clock}
        \item Definição de circuitos e operações simples
        \item Hoje, muito já vem pré-definido
    \end{itemize}

    High-Level Synthesis (\alert{HLS}):
    \begin{itemize}
        \item Gerar HDL a partir de \alert{código em C}
        \item \alert{OpenCL}
    \end{itemize}
\end{frame}

\begin{frame}
    \frametitle{FPGAs: Saiba Mais}
    Livros:

    \begin{itemize}
        \item Andrew Moore and Ron Wilson. FPGAs for Dummies. Intel/ Wiley, 2017
        \item Maxfield, Clive. The design warrior's guide to FPGAs: devices, tools and flows. Elsevier, 2004
        \item Vanderbauwhede, Wim, and Khaled Benkrid, eds. High-performance computing using FPGAs. New York: Springer, 2013 (Avançado)
    \end{itemize}
\end{frame}

% \section{GPGPUs}
%
% \subsection{História e Computação de Propósito Geral}
%
% \begin{frame}
%     \frametitle{GPUs: História e Computação de Propósito Geral}
%     \begin{center}
%         \includegraphics[width=\textwidth]{spiderdoom}
%     \end{center}
% \end{frame}
%
% \frame{
% \frametitle{Placas de Vídeo}
%     \begin{minipage}{0.5\textwidth}
%         \begin{itemize}
%             \item 80's: Primeiro controlador de vídeo
%             \item Evolução dos jogos 3D
%             \item Aplicação de texturas, iluminação, sombras, $\dots$
%             \item Animação, simulação computacional, design gráfico, física da luz, $\dots$
%         \end{itemize}
%     \end{minipage}\hfil
%     \begin{minipage}{0.45\textwidth}
%         \begin{figure}[htpb]
%         \centering
%         \includegraphics[scale=.175]{Doom-screenshots}\vspace{0.5cm}
%         \includegraphics[scale=.135]{Call-of-Duty-Black-Ops}
%         \label{fig:games3D}
%         \end{figure}
%     \end{minipage}
% }
%
% \frame{
%     \frametitle{Unidades de Processamento Gráfico (GPUs)}
% \small
% \begin{minipage}{0.6\textwidth}
%     \begin{itemize}
%         \item O termo \alert{GPU} foi popularizado pela Nvidia em 1999, quando criou a GeForce 256
%         \item Em 2002 foi lançada a primeira GPU de Propósito Geral (\alert{GPGPU})
%         \item Os principais fabricantes de GPUs são a \alert{Nvidia} e a \alert{AMD}
%         \item Em 2005 a Nvidia lançou \alert{CUDA}
%         \item Inteligencia Artificial, Carros autônomos, Realidade Virtual, $\dots$
%     \end{itemize}
% \end{minipage}
% \hfil
% \begin{minipage}{0.35\textwidth}
%     \begin{figure}[htpb]
%         \centering
%         \includegraphics[scale=.3]{Geforce-256}
%         \label{fig:GeForce-256}
%     \end{figure}
% \end{minipage}
% }
%
% \frame{
% \frametitle{GPU vs. CPU}
% \begin{block}{}
% Hoje em dia, GPUs são capazes de fazer em paralelo mais operações computacionais que CPUs multi-core
% \end{block}
% \begin{figure}[h!]
% \centering
% \includegraphics[scale=.175]{gflops}\hfill
% \includegraphics[scale=.175]{bandwidth}
% \label{fig:gflops}
% \end{figure}
% }
%
% \frame{
%     \frametitle{GPU de Propósito Geral (GPGPU)}
% O programa principal é executado na CPU (\alert{host}), que é responsável por
% iniciar a execução na GPU (\alert{device}).
%
% As GPUs têm sua própria herarquia de memória e os dados devem ser transferidos
% através do barramento \alert{PCI Express}.
%
% \centering
% \includegraphics[scale=.75]{gpu-computing}
% }
%
% \frame{
% \frametitle{Supercomputadores com Aceleradores de Hardware}
% Supercomputadores na Top 500 com aceleradores e co-processadores:
%
% \begin{figure}[htpb]
% \centering
% \includegraphics[scale=.225]{top-500-accelerators}
% \label{fig:top500Acc}
% \end{figure}
% }
%
% \frame{
% \frametitle{Supercomputadores e Consumo de Energia}
% \alert{Green 500}: Lista de supercomputadores mais eficientes em consumo de energia
%
% \begin{figure}[htpb]
% \centering
% \includegraphics[scale=.36]{Green500Top10}
% \label{fig:Grentop500}
% \end{figure}
% }
%
% \subsection{GPGPUs Nvidia}
%
% \begin{frame}
%     \frametitle{GPGPUs Nvidia}
%     \begin{center}
%         \includegraphics[width=\textwidth]{gtx1080}
%     \end{center}
% \end{frame}
%
% \frame{
% \frametitle{Roadmap para Arquiteturas de GPUs NVIDIA}
% A próxima arquitetura será a Volta, com 12-10 nm FinFET.
% \begin{figure}[htpb]
%  \centering
% \includegraphics[scale=0.3]{GPU-Roadmap-GTC-2015-HGEMM}
% \end{figure}
% }
%
% \frame{
% \frametitle{Espaços de Memória em GPGPUs}
% O acesso à memória global é custoso, a latência é $100x$ mais lenta que na memória compartilhada.
% \begin{figure}[htpb]
%  \centering
% \includegraphics[scale=0.6]{memory-spaces-on-cuda-device}
% \caption{Espaços de memória em dispositivos CUDA}
% \end{figure}
% }
%
% \frame{
% \frametitle{Compute Capability de GPUs NVIDIA}
% Compute Capability é uma diferenciação entre arquiteturas e modelos de GPUs NVIDIA.
% \begin{figure}[htpb]
%  \centering
% \includegraphics[scale=0.2]{gpuArchMap}
% \end{figure}
% }
%
% \frame{
% \frametitle{GPUs Tesla da NVIDIA}
% \begin{minipage}{0.4\textwidth}
%     \begin{itemize}
%         \item Multiprocessadores
%         \item FP 32
%         \item FP 64
%         \item Interface mem.
%         \item TDP
%         \item Fabricação
%     \end{itemize}
% \end{minipage}\hfill
% \begin{minipage}{0.6\textwidth}
% \begin{figure}[htpb]
%  \centering
% \includegraphics[scale=0.2]{Teslas-GPUs}
% \label{fig:TeslasGPUs}
% \end{figure}
% \end{minipage}
% }
%
% \frame{
% \frametitle{Arquitetura Tesla: Primeiras GPGPU}
% \begin{figure}[htpb]
%  \centering
% \includegraphics[scale=0.6]{Geforce-256-arch}
% \label{fig:Teslaarch}
% \end{figure}
% }
%
% \frame{
% \frametitle{Multiprocessador da Arquitetura Fermi}
% \scriptsize
% \begin{minipage}{0.35\textwidth}
%     \begin{itemize}
%         \item Processo de fabricação 40 nm
%         \item 16 Multiprocessadores
%         \item 32 processadores
%         \item 16 unidades leitura/escrita
%         \item 4 unidades funções especiais
%         \item 48 kB Memória compartilhada
%         \item 32768 registradores
%         \item 2 escalonadores de warps
%     \end{itemize}
% \end{minipage}\hfill
% \begin{minipage}{0.65\textwidth}
% \begin{figure}[htpb]
%  \centering
% \includegraphics[scale=0.265]{fermi-sm}
% \label{fig:FermiSM}
% \end{figure}
% \end{minipage}
% }
%
% \frame{
% \frametitle{Arquitetura Fermi}
% \begin{figure}[htpb]
%  \centering
% \includegraphics[scale=0.275]{fermi-arch}
% \label{fig:Fermiarch}
% \end{figure}
% }
%
% \frame{
% \frametitle{Multiprocessador da Arquitetura Kepler}
% \scriptsize
% \begin{minipage}{0.35\textwidth}
%     \begin{itemize}
%         \item Processo de fabricação 28 nm
%         \item 15 SMX de 192 cores
%         \item 2880 CUDA Cores
%         \item 64 KB do arquivo de registradores
%         \item 3072 KB de cache L2
%         \item Até 24 GB de memória GDDR5
%         \item Hyper-Q
%         \item Paralelismo dinâmico
%         \item 4 escalonadores de warps
%     \end{itemize}
% \end{minipage}\hfill
% \begin{minipage}{0.65\textwidth}
% \begin{figure}[htpb]
%  \centering
% \includegraphics[scale=0.2]{kepler-smx}
% \label{fig:keplerSM}
% \end{figure}
% \end{minipage}
% }
%
% \frame{
% \frametitle{Arquitetura Kepler}
% \begin{figure}[htpb]
%  \centering
% \includegraphics[scale=0.2]{kepler-gk110-arch}
% \label{fig:keplerarch}
% \end{figure}
% }
%
% \frame{
% \frametitle{Multiprocessador da Arquitetura Maxwell}
% \begin{minipage}{0.6\textwidth}
% \scriptsize
%     \begin{itemize}
%         \item Cluster Processamento Gráfico-GPC
%         \item 128 CUDA cores por SMM
%         \item Dividido em 4 distintos 32 cores FP
%         \item 96 KB de memória compartilhada dedicada
%         \item 48 KB de cache L1/cache textura (unificado)
%         \item 2 MB  de memória cache L2
% \end{itemize}
% \end{minipage}\hfill
% \begin{minipage}{0.4\textwidth}
%
% \begin{figure}[htpb]
%  \centering
% \includegraphics[scale=0.3]{maxwell-smm}
% \label{fig:maxwellSMM}
% \end{figure}
% \end{minipage}
% }
%
% \frame{
% \frametitle{Arquitetura Maxwell}
% \begin{figure}[htpb]
%  \centering
% \includegraphics[scale=0.325]{maxwell-gpc}
% \label{fig:maxwellarch}
% \end{figure}
% }
%
%
% \frame{
% \frametitle{Arquitetura Pascal}
%     \begin{itemize}
%         \item Cada SM está dividido em 2 blocos de processamento
%         \item SMs of 64 cores de precisão simples
%         \item 32 unidades de precisão dupla
%         \item 64/2 KB de memória compartilhada dedicada
%         \item 4096-bit na interface de memória
%         \item HBM Stacked DRAM
%         \item Preempção na computação
%         \item NVLink
%     \end{itemize}
% }
%
%
% \frame{
% \frametitle{Multiprocessador da Arquitetura Pascal}
% \begin{figure}[htpb]
%  \centering
% \includegraphics[scale=0.425]{pascal-sm}
% \caption{Multiprocessador da arquitetura Pascal}
% \label{fig:pascalSMM}
% \end{figure}
% }
%
%
% \frame{
% \frametitle{Arquitetura Pascal: NVLink}
% \begin{figure}[htpb]
%  \centering
% \includegraphics[scale=0.9]{Nvlink.png}
% \caption{NVLink ligando 8 Tesla P100 em uma topologia Hibrida Cubo malha}
% \label{fig:nvlink}
% \end{figure}
% }
%
% \subsection{GPGPUs AMD}
%
% \begin{frame}
%     \frametitle{GPGPUs AMD}
%     \begin{center}
%         \includegraphics[width=\textwidth]{r9390p}
%     \end{center}
% \end{frame}
%
% \frame{
% \frametitle{GPUs de Próposito Geral da AMD}
% Evolução das GPUs AMD:
%
% \begin{itemize}
%     \item Array Technology Inc. (ATI): 1985
%     \item Visual Processing Unit (VPU), da ATI
%     \item Advanced Micro Devices (AMD) adquire a ATI: 2006
%     \item Graphic Core Next (GCN): 2011
%     \item AMD Radeon HD 7790, usa GCN 1.1: 2012
%     \item GCN: 5a. Geração
% \end{itemize}
% }
%
% \frame{
%     \frametitle{Graphic Core Next}
% \begin{minipage}{0.475\textwidth}
%     \begin{figure}[htpb]
%         \centering
%         \includegraphics[scale=0.225]{AMD-GCN-3Th}
%         \label{fig:AMD-GCN}
%     \end{figure}
% \end{minipage}\hfill
% \begin{minipage}{0.475\textwidth}
%     \begin{figure}[htpb]
%         \centering
%         \includegraphics[scale=0.25]{Core-AMD-GCN}
%         \label{fig:Core-AMD-GCN}
%     \end{figure}
% \end{minipage}
% }
%
% \frame{
% \frametitle{Roadmap para as GPUs AMD}
% \begin{figure}[htpb]
%  \centering
% \includegraphics[scale=0.45]{AMDGPU}
% \label{fig:roadmapAMD}
% \end{figure}
% }
%
% \section{Intel Xeon Phi}
%
% \begin{frame}
%     \frametitle{Intel Xeon Phi}
%     \begin{center}
%         \includegraphics[width=\textwidth]{xeonphi}
%     \end{center}
% \end{frame}
%
% \frame{
% \frametitle{Roadmap para Intel Xeon Phi}
% \begin{itemize}
% \item Larrabee: 2006-2010 $\rightarrow$ Knights Ferry: 2010
% \item 1$^a$. geração x100 Knights Corner: 22 nm
% \item 2$^a$. geração x200 Knights Landing: 14 nm
% \item 3$^a$. geração Knights Hill: 10 nm
% \end{itemize}
%
% \begin{figure}[h!]
% \centering
% \includegraphics[scale=.65]{roadMapKnightIntel}
% \label{fig:RMXeonphi}
% \end{figure}
% }
%
% \frame{
% \frametitle{Knights Corner: Arquitetura}
% \begin{figure}[h!]
% \centering
% \includegraphics[scale=.475]{intelXeonPhi}
% \label{fig:Xeonphi}
% \end{figure}
% }
%
% \frame{
% \frametitle{Anel de Interconexão Bidirecional}
% \begin{figure}[h!]
% \centering
% \includegraphics[scale=.6]{bidirectionalRingXeonPhi}
% \label{fig:AnelBidireccional}
% \end{figure}
% }
%
% \frame{
% \frametitle{Core do Xeon Phi}
% \begin{figure}[h!]
% \centering
% \includegraphics[scale=.475]{CoreXeonPhi}
% \label{fig:coreXeonPhi}
% \end{figure}
% }
%
% \frame{
% \frametitle{Unidade de Processamento Vetorial}
% \begin{figure}[h!]
% \centering
% \includegraphics[scale=.475]{vectorProcessingUnitXeonPhi}
% \label{fig:VPUXeonPhi}
% \end{figure}
% }
%
% \frame{
% \frametitle{Desempenho por Watts de alguns aceleradores}
% \begin{figure}[h!]
% \centering
% \includegraphics[scale=.475]{PerfWattsXeonPhi}
% \label{fig:PerfWattsXeonPhi}
% \end{figure}
% }
%
% \frame{
% \frametitle{Intel Parallel Studio}
% \begin{figure}[h!]
% \centering
% \includegraphics[scale=.4]{Intel-VTune}
% \caption{Intel Vtune Amplifier}
% \label{fig:Vtune}
% \end{figure}
% }
%
% \frame{
% \frametitle{OpenACC}
% OpenACC:
% \begin{itemize}
%     \item Padrão para Programação Paralela
%     \item Baseado no compilador PGI (Portland Group)
%     \item Fornece coleção de diretivas para especificar laços e regiões de
%         código paralelizáveis
% \end{itemize}
%
% \begin{figure}[htpd]
% \centering
% \includegraphics[scale=.4]{accelerators}
% \end{figure}
% }
%
% \maketitle

\end{document}
